\section{Introduction}

Guided by community-wide input, the LSST is designed to achieve multiple goals in four main
science themes: Taking an Inventory of the Solar System, {\bf Mapping the Milky Way}, Exploring the
Transient Optical Sky, and Probing Dark Energy and Dark Matter. These are just four of the many
areas on which LSST will have enormous impact, but they span the space of technical challenges
in the design of the system and the survey and have been used to focus the science requirements.


Encoded in the structure, chemical composition and kinematics of stars in our Milky Way is
a history of its formation. Surveys such as 2MASS and SDSS have demonstrated that the
halo has grown by accretion and cannibalization of companion galaxies, and it is clear that
the next steps require deep wide-field photometry, parallax, proper motions, and spectra to
put together the story of how our Galaxy formed. LSST will enable studies of the distribution of 
numerous main sequence stars beyond the presumed edge of the Galaxy?s halo,
their metallicity distribution throughout most of the halo, and their kinematics beyond the
thick disk/halo boundary, and will obtain direct distance measurements below the hydrogen burning limit for a representative thin-disk sample. 
LSST is ideally suited to answering two
basic questions about the Milky Way Galaxy: What is the structure and accretion history of
the Milky Way? What are the fundamental properties of all the stars within 300 pc of the
Sun?


LSST will produce a massive and exquisitely accurate photometric and astrometric data set.
Compared to SDSS, the best currently available optical survey, LSST will cover an area more
than twice as large, using hundreds of observations of the same region in a given filter instead
of one or two, and each observation will be about two magnitudes deeper. LSST will detect
of the order 1010 stars, with sufficient signal-to-noise ratio to enable accurate light curves,
geometric parallax, and proper motion measurements for about a billion stars. Accurate
multi-color photometry can be used for source classification (1\% colors are good enough to
separate main sequence and giant stars, Helmi et al. 2003), and measurement of detailed
stellar properties such as effective temperatures to an rms accuracy of 100 K and metallicity
to 0.3 dex rms.

To study the metallicity distribution of stars in the Sgr tidal stream (Majewski et al. 2003)
and other halo substructures at distances beyond the presumed boundary between the inner
and outer halo ($\sim$30 kpc, Carollo et al. 2007), the coadded depth in the $u$ band must reach
$\sim$24.5. To detect RR Lyrae stars beyond the Galaxy?s tidal radius at $\sim$300 kpc, the s
ingle visit depth must be $r\ sim $24.5. In order to measure the tangential velocity of stars to an
accuracy of 10 km s$^{?1}$
at a distance of 10 kpc, where the halo dominates over the disk, proper
motions must be measured to an accuracy of at least 0.2 mas yr$^{?1}$.

The same accuracy follows
from the requirement to obtain the same proper motion accuracy as Gaia (Perryman et al.
2001) at its faint limit ($r \sim$ 20). In order to produce a complete sample of solar neighborhood
stars out to a distance of 300 pc (the thin disk scale height), with 3$\sigma$ or better geometric
distances, trigonometric parallax measurements accurate to 1 mas are required. To achieve
the required proper motion and parallax accuracy with an assumed astrometric accuracy of
10 mas per observation per coordinate, approximately 1,000 observations are required. This
requirement on the number of observations is close to the independent constraint implied by
the difference between the total depth and the single visit depth.



\subsection{Science Collaboration Structure}
 
Our collaboration is organized into 7 Working Groups, each of which reflects a core scientific area being worked on by collaboration members. All collaboration members belong to at least one Working Group, and in early 2013, each Working Group was charged with creating a brief roadmap that highlights technical and scientific challenges that must be worked through to conduct our science with LSST. These roadmaps have been distilled into a list of active priorities, and the ongoing work tab gives a few examples of action items that collaboration members are actively pursuing.


Working Groups \& Leads:
\begin{enumerate}
\item The Solar Neighborhood: Adam Burgasser \& Todd Henry, lsst-milkyway-low@lsstcorp.org
\item Star Clusters: Kevin Covey \& Jay Strader, lsst-milkyway-sc@lsstcorp.org
\item The Galactic Bulge: Will Clarkson \& Victor Debattista, lsst-milkyway-bulge@lsstcorp.org
\item Galactic Structure and ISM: Peregrine McGehee, lsst-milkyway-struct@lsstcorp.org
\item Magellanic Clouds: Knut Olsen, lsst-milkyway-mc@lsstcorp.org
\item Near Field Cosmology: Marla Geha \& Carl Grillmair, lsst-milkyway-nfc@lsstcorp.org
\item Variable Stars: David Ciardi, lsst-milkyway-var@lsstcorp.org
\end{enumerate}

\subsubsection{The Solar Neighborhood}

Key target populations:
\begin{itemize}
\item{Solar 
neighborhood: classically d $<$ 10-25 pc, extend to $\sim$100 pc in LSST era}
\item{Young 
stars: clusters, moving groups \& associations }
\item{Old 
stars: subdwarfs, thick disk and halo objects, globular clusters (??) }
\item{Brown 
dwarfs: $<$0.07 Msun, including free-floating planets }
\item{White 
dwarfs (and other remnants?) }
\item{Binaries 
and higher order multiples: q $>$ 0.1, resolved and unresolved}
\item{Low 
mass companions: q $<$ 0.1, resolved and unresolved }
\end{itemize}


\subsubsection{Star Clusters}

Star clusters are a critical component of many of the science goals listed in the LSST Science Book (Ivezic et al. 2010). They offer unique environments where stars with a range of masses but similar ages and chemical compositions can be found. One class of studies aims to leverage these relatively simple stellar populations to understand the dependence of the stellar mass function and stellar evolution on metallicity and age (e.g., Sec. 6.5-6.6, 6.9-6.11; LSST Sci. Book). Their known properties and distances also motivate calibration work on many kinds of variable stars and transients (e.g., Sec 8.9; LSST Sci. Book). 

A separate class of studies is focused on using star clusters to trace the star formation history, chemical evolution, and galactic structure of the Milky Way, nearby galaxies (including the Magellanic Clouds), and even more distant galaxies extending beyond the Local Volume (e.g., Sec. 6.2-6.3, 6.6, 7.2, 7.3, 7.8, 7.10, 7.11; LSST Sci Book). 

\subsubsection{The Galactic Bulge}

These are kept deliberately broad at this stage (see Section 7 for more information), but point to the unique strengths of the Bulge as a test-case for Galaxy formation models: 
\begin{enumerate}
\item Detailed balance of populations as traced by morphology, chemistry, kinematics and distance 
\item The present-day mass distribution of the inner Milky Way as traced by Bulge star motions 
\item Wide-field examinations of other sub-populations and objects of interest. 
\end{enumerate}

\subsubsection{Galactic Structure and ISM}

{\bf Nothing!}

\subsubsection{Magellanic Clouds}

The LMC and SMC are the nearest large galaxies to the Milky Way and represent  fundamental testbeds for investigations related to many stellar astrophysical topics,  from studying the star formation process, to the physics of variable star populations,  to the physics of galaxy interactions, and to calibrating the cosmological distance  scale.    

The Clouds are some of the bestGstudied galaxies in the Universe, yet nevertheless  consistently have the power to surprise.  A recent surprise is that the Clouds, which  have long been thought of as closely bound satellites of the Milky Way, appear to be  on first infall, or at most have made a small number of close passages, based on  proper motion measurements (Kallivayalil et al. 2006a,b; Besla et al. 2007, Piatek et  al. 2008, Kallivayalil et al. 2013).  We have also seen evidence that the LMC has  accreted a significant mass in stars from the SMC (Olsen et al. 2011), which may  indicate that the LMC and SMC collided directly with each other ~100 Myr ago  (Besla et al. 2012), a discovery that relied on the proper motion work.  The HI tidal  streamers that connect the LMC to the SMC and Magellanic Stream are kinematically  linked to the accreted SMC stars, and thus may have formed in this collision; the  streamers converge on the location of 30 Doradus in the LMC (Nidever et al. 2008),  indicating that 30 Doradus may be an interactionGinduced star formation event.   Moreover, the stellar debris left behind by the SMC could naturally explain the  microlensing event rate (Besla et al. 2013) seen towards the LMC by e.g. MACHO and  OGLE.  

The Magellanic Clouds are also now known to extend to much larger radius than  known before, through several pencilGbeam studies.  Majewski et al. (2009) found  spectroscopicallyGconfirmed KGgiant LMC stars at large distances (R $\sim$ 22 or 19 kpc),  while Saha et al. (2010) found that the LMC exponential disk extends past 12 disk  scale lengths.  In the SMC, Nidever et al. (2011) traced K giants to $\sim$11, with  possible evidence for a stellar halo.  These studies are based on exploration of 1  of the relevant area of sky.  There is thus a vast area that remains to be explored for  hidden Magellanic Clouds structure, with implications both for understanding the  history of the Clouds and the formation of the Galactic halo.   

There are many other recent exciting discoveries in the Magellanic Clouds that have  not been mentioned here.  The point of mentioning the few above is to demonstrate  that 1) in the Clouds, astrophysical problems are often connected across broad areas  of study, 2) Magellanic Cloud populations cover a very large area of sky, and 3)  highly precise measurements, such as produced by the work on LMC and SMC  proper motions, have the power to fundamentally alter accepted understanding.  

\subsection{Cross-collaboration initiatives}

\subsubsection{Variable Stars}

The science of variable stars is effectively split between the Transients \& Variable Stars ( TVS) Work- ing Group and the Stars, Milky Way, and Local Volume (SMWLV) Working Group. This subgroup will serve as a bridge between the two Working Groups to maintain a focus on this important scientific area. 

{\bf Collaboration with Transients \& Variable Stars Working Group:}
One of the key questions about our subgroup is what science will be primarily covered by the SMWLV group versus the TVS group. In general, extragalactic transients like supernovae and GRBs will be dealt with by the TVS. That still leaves eclipsing, rotational, pulsating, and flaring stars. So far most of the work on pulsating stars (i.e. RR Lyrae) has taken place within the TVS group, and so we will only briefly discuss such variables in this roadmap. However, we should remain careful to prevent any scientifically interesting variable type from falling through the cracks between the two working groups. 

{\bf Classification of Variables:}
A very broad issue that both Working Groups (TVS and SMWLV) will have to deal with is the nature of LSST classification. No science can be accomplished in the area of variable stars unless we resolve the question of how the LSST pipeline will classify the enormous variety of variable stars, and at what level. 
Will stars be simply identified as Variable or Not Variable based on some confidence level of a given statistical test? Or will the classification go deeper, into identifying stars as periodically versus non- periodically variable, or even further into sorting them into specific categories, such as eclipsing, pulsating, etc.? This is arguably the single most important question for variable star science with the LSST. There are existing efforts to approach this problem, including the EB Factory project (Stassun et al. 2013) at Vanderbilt University1, and efforts led by Josh Bloom (Bloom et al. 2012). However those efforts will have to be more fully developed for the variable star science goals of LSST. Current variability surveys such as the Palomar Transient Factory and the Catalina Sky Survey offer precursor data sets that can be used for testing classification methods. 

\subsubsection{Near Field Cosmology}

Dark matter constitutes roughly 85\% of the matter density of the Universe, and represents a critical gap in our understanding of fundamental physics. Despite these extensive experimental efforts, the only robust, positive empirical measurement of dark matter continues to come from cosmological and astrophysical observations. The Large Synoptic Survey Telescope (LSST) offers a unique avenue to attack the dark matter problem. Our group seeks to identify and pursue scientific avenues to utilize LSST to help us understand the fundamental physics that governs dark matter. Specifically, we hope to identify the fundamental constituents of dark matter (e.g., new fundamental particles, fields, or compact objects) and to characterize the properties of these constituents (e.g, mass, temperature, self-interaction rate).
A major initiative in the study of dark matter has been formed with the LSST Dark Energy Science Collaboration [DESC], see
{\it https://lsstdarkmatter.github.io/index.html}.

The LSST Near Field Cosmology subgroup focuses on using stellar substructure throughout the Milky Way
 and Local Volume as a probe of the underlying CDM cosmology.  
This includes searches for dwarf galaxies, tidal streams and other coherent features in 
phase space using the properties of individual stars. 
Examples include searches for ultra-faint galaxies based on overdensity of main sequence stars, 
searches for substructure using RR Lyrae variables, or searches for moving stellar groups using proper motions. 

Searches for Milky Way substructure will allow critical new constraints in several areas. 
A more complete census and map of the distribution of satellites around the Milky Way sheds 
light both on the formation of satellite systems and the nature of dark matter halos that are 
able to host such low-mass galaxies.
 Finding tidal debris beyond 100 kpc and out to the virial radius of the Milky Way will open a 
window on the more recent accretion history of the Galaxy, as well as the accretion of objects 
on more radial orbits in particular. 
Debris in the outer Galaxy also promises a powerful new probe of the mass 
distribution around the Milky Way in regions of a dark matter halo that have been largely 
unexplored for any galaxy in the Universe. 

\subsubsection{Education \& Public Outreach}

{\bf Needs text}

