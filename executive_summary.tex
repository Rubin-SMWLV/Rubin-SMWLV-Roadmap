\section{Executive Summary}

The Large Synoptic Survey Telescope (LSST) is an 8.4-m dedicated survey telescope with a 3.2 Gigapixel camera under construction by the US National Science Foundation and the US Department of Energy. Construction of the camera is managed by the SLAC National Accelerator Laboratory, while the LSST observatory is under construction on its mountain-top site at Cerro Pachon, Chile. Work toward meeting the LSST science goals is being carried out by the LSST Science Collaborations.

The Stars, Milky Way and Local Volume Collaboration has the overarching goals of understanding the accretion history and structure of the Milky Way and the Local Volume, and the fundamental properties of stars within 300 pc of the Sun. These are also some of the main science drivers of both the telescope and the survey design of LSST, which will have unprecedented capability in the faint time domain.

