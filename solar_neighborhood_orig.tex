\section{LSST Solar Neighborhood Collaboration Roadmap Planning Document }

Adam Burgasser, John Gizis, Todd Henry, Sebastien Lepine, Michael Liu, Keivan Stassun 



\subsection{Purpose of this Document }
\begin{itemize}
\item{Identify 
the technical and scientific challenges for LSST in studying low mass stars and  the solar neighborhood }
\item{Identify 
tasks for collaboration members that can overcome these challenges on a  timescale of a few years. }
\item{Organize 
collaboration members to perform assigned tasks before data delivery }
\end{itemize}


\subsection{History }

\begin{itemize}
\item 22 October 2013 (AJB): Version 1 released to working group 
\item 1 November 2013 (AJB): comments integrated and sent to leads 
\item 2 January 2013 (AJB): comments from WIllman 
\item 14 August 2014 (MCL): cleanup of edits 
\end{itemize}

\subsection{Introductory Material}

\subsubsection{Key target populations:}
\begin{itemize}
\item{Solar 
neighborhood: classically d $<$ 10-25 pc, extend to $\sim$100 pc in LSST era}
\item{Young 
stars: clusters, moving groups \& associations }
\item{Old 
stars: subdwarfs, thick disk and halo objects, globular clusters (??) }
\item{Brown 
dwarfs: $<$0.07 Msun, including free-floating planets }
\item{White 
dwarfs (and other remnants?) }
\item{Binaries 
and higher order multiples: q $>$ 0.1, resolved and unresolved}
\item{Low 
mass companions: q $<$ 0.1, resolved and unresolved }
\end{itemize}

\subsubsection{Definitions \& Acronyms used in the text}
Mass scales: 
\begin{itemize}
\item{$>$ 
2 Msun = “massive” }
0.5-2 
\item{
Msun = “solar type” }
\item{0.1-0.5 
Msun = “low mass” (LM) }
\item{$<$ 
0.1 = “very low mass” (VLM) }
\item{$<$ 
0.07 Msun = “brown dwarf” (BD) }
\item{$<$ 
0.013 Msun = “planetary-mass object” (PMO) }
\end{itemize}


CPM = common proper motion  DBMM = deuterium burning minimum mass, nominally 0.013 Msun for solar Z  GC = globular cluster  HBMM = hydrogen burning minimum mass, nominally 0.07 Msun for solar Z  LBMM = lithium burning minimum mass, nominally 0.06 Msun for solar Z 
[M/H] = metallicity 
MF = mass function  MG = moving group  MS = main sequence star  NIR = near infrared  PM = proper motion  PMS = pre-main sequence star  RPM = reduced proper motion  RV = radial velocity 
 SED = spectral energy distribution  SpT = spectral type  ToO = target of opportunity  WD = white dwarf  YA = young association  YMG = young ($<$100 Myr) moving group 

\subsection{Roadmap Plan }
\begin{itemize}
\item{Phase 
1 (deadline: 1 Nov 2013): Highlight technical/scientific challenges that must be  worked on to conduct our science with LSST. }
\item{Phase 
2 (deadline: TBD): Develop a strawman path towards overcoming those  challenges before LSST data start flowing. }
\item{ Phase 
3 (deadline: TBD): raise interesting precursor science that can be done by  you/can get grant support now. }
\end{itemize}


\subsection{Identified inputs required for planning (priorities TBD)}
\begin{itemize}
\item{precise 
LSST filter definitions }
\item{photometric 
sensitivity per band for single and deep pointing }
\item{astrometric 
performance (proper motion \& parallax) per band as a function of 
brightness, spectral type, \# of epochs, time baseline and sky location (see preliminary 
results from DAWG = Differential Astrometry Working Group) }
\item{LSST 
imaging cadence(s) }
\item{LSST 
PSF shape model (including saturation) }
\item{galactic 
model }
\item{ spectroscopic 
followup resources }
\item{theoretical 
atmospheric models of LMs/VLMs/BDs as a function of {Teff, log g, [M/H], 
clouds, circulation} that span LSST bands }
\item{empirical 
templates of sources (MLTY standards, subdwarfs and young dwarfs) }
\item{evolutionary 
models and forward modeling samples for BD population }
\item{flare 
emission model to determine LSST response/sensitivity to LM flares }
\item{cloud 
spot model to determine LSST response/sensitivity to BD weather patterns }
\item{An 
(updateable) list of currently nearby LMs/VLMs/BDs (ra, dec, spt, distance, predicted 
LSST mags, age/cluster, etc) }
\item{Quantification 
of completeness as a function of SpT and distance }
\end{itemize}

\subsection{Primary science investigations: Objectives, issues and tasks}



\subsubsection{Volume-complete (astrometric) sample of extended solar neighborhood }

definition: all sources for which parallax measurements can be made with LSST  

interested collaborators: 
Adam Burgasser,
John Gizis,
Todd 
Henry,
Michael 
Liu,
Keivan 
Stassun,
Sebastien 
Lepine.

interface with other collaborations: 
Galactic Structure and ISM: 

objectives:
\begin{itemize}
\item distance-limited sample of solar neighborhood (to be defined) 
\item 6D phase space determination (XYZUVW) 
\item LM/VLM/BD MF determination 
\item Full physical, atmospheric, multiplicity characterization of complete sample  
\item map evolution of ultracool objects through the M/L, L/T, and T/Y transitions. 
\end{itemize}


issues: 
\begin{itemize}
\item number of epochs and cadence required to distinguish proper motion and parallax 
\item overlap with GAIA (minimal/non-existent for L/T/Y dwarfs) 
\item astrometric calibration and astrometric uncertainties 
\item distance limit for completeness 
\item brightness limit for completeness (e.g., astrometric accuracy vs. brightness) 
\item detection of/contamination by astrometric binaries and unresolved binaries 
\item followup spectroscopy: science goals \& data needs (sample size, resolution, etc.) 
\item RV measurements 
\item kinematics (statistical) ages? 
\item handling crowding along Galactic plane (e.g. image differencing, pixel-level modeling) 
\end{itemize}

tasks to do: 
\begin{itemize}
\item{need 
a current list of stars (perhaps broken down by SpT) w/in XX pc to assess where the 
big holes in completeness are - both X and SpT regime that we care about will need to  be defined (Todd?), will require some “population synthesis” modeling for BDs (Adam?),  as well as kinematic considerations (Paul Thorman?) }
\item{summarize 
existing compliations from Pan-STARRS solar neighborhood work (Mike) }
\item{model 
astrometric binary contribution/contamination }
\item{comprehensive 
follow-up of sources - RV, high resolution imaging }
\item{examine 
model parameter sensitivity through LSST photometry }
\item{Lists 
of proper motion selected M dwarfs already exist to at least 100 pc. A local  kinematics model (velocity space distribution), combined with luminosity function, will  determine how many stars are missing from the current census as a function of  magnitude, and e.g. what the distance range of GAIA will be for stars of different  magnitudes. This will determine the “sweet spot” (in distance/luminosity) for LSST to  reap the most results. }
\end{itemize}


tasks completed: 

\subsubsection{Deep sample of LM/VLM/BD dwarfs}

definition: sources inaccessible to parallax  

interested collaborators:  

interface with other collaborations: 
Galactic Structure and ISM: 
The Galactic Bulge: 
Near Field Cosmology: 

objectives: 
\begin{itemize}
\item{extremely 
large sample ($10^6$?) for population analysis }
\item{galactic 
stratigraphy - chemical abundances, kinematic “heating”, structural features 
out to XX kpc }
\item{galactic 
scaleheight as a function of mass (constraint on potential and formation 
mechanisms)} 
\item{search 
for thick disk/halo objects }
\end{itemize}

issues: 
\begin{itemize}
\item{spectroscopy 
won’t be viable for whole sample - photometric selection criteria, RPM 
(limits) }
\item{most 
sensitive filter likely to be Y(4): will single-band detections(+astrometry) be useful? }
\item{limit 
to which PM is viable }
\item{metallicity 
characterization through photometry alone }
\end{itemize}

tasks to do: 
\begin{itemize}
\item{simulate 
yield along different sitelines, with variations in scale height, MF, etc. }
\item{calibration 
sample for metallicity effects  tasks completed: }
\end{itemize}


\subsubsection{Ultracold brown dwarfs (late-T and Y dwarfs)} 
definition: BDs cooler than $\sim$1000 K => T5 and later, any distance  

interested collaborators: 
Adam Burgasser, Michael Liu 

interface with other collaborations: 
Galactic 
Structure and ISM: 

objectives: 
\begin{itemize}
\item{detect 
a significant population of T and Y dwarfs for MF determination }
\item{kinematic 
characterization (6D phase space) }
\item{identify 
extreme TY populations (halo, young, cloudy, etc.) }
\item{measure 
how optical SEDs change across K$->$KCl, Na$->$NaCl/Na2S transitions }
\end{itemize}

issues: 
\begin{itemize}
\item{sensitivity 
of LSST filters to cool T and Y dwarfs - detection limits, expected numbers 
assuming an MF }
\item{compare 
expected performance to WISE. 
e.g. LSST will not be competitive in this area, 
at least for discovery of the coolest objects. }
\item{most 
sensitive filter likely to be Y(4): will single-band detections(+astrometry) be useful? }
\item{possible 
to spectroscopically follow these up? (necessary for classification, RV) }
\item{minimal 
information set required to eliminate contaminants (e.g., RPM, color sets, 
necessary filters) }
\end{itemize}

tasks to do: 
\begin{itemize}
\item{examine 
photometric classification and contamination}
\end{itemize}

tasks completed: 


\subsubsection{Halo dwarfs across the substellar limit}

definition: dwarfs with halo kinematics and/or $[M/H] \le -1$  

interested collaborators: 
Pat Boeshaar,  Adam Burgasser, John Gizis,  Michael Liu,
Sebastien Lepine 

interface with other collaborations: 
Galactic 
Structure and ISM: 

objectives: 
\begin{itemize}
\item{Halo 
mass function across substellar limit}
\item{measurement 
of H-burning “age gap” (z-dependent HBMM)}
\item{confirmation 
of BD formation in low metallicity environment}
\item{metallicity 
effects on low mass spectra/photometry }
\item{multiplicity 
of halo LM/VLM/BD dwarfs }
\item{map 
out velocity space distribution }
\item{search 
for substructure in LM halo }
\item{discovery 
of extremely metal poor ($[Fe/H] < -3$) VLM dwarfs }
\item{Pop 
III VLM dwarfs? }
\end{itemize}

issues: 
\begin{itemize}
\item{game 
selection: target max V (anticenter and vertical ring) }
\item{efficient 
and robust selection: RPM, metallicity effects, contaminants}
\item{mapping 
LSST photometry to physical properties - templates? models?}
\item{spectral 
characterization? }
\end{itemize}

tasks to do: 
\begin{itemize}
\item{model 
discovery rate using MF forward modeling + atmosphere models (colors) }
\item{identify 
templates for SED characterization }
\item{estimate 
the local density of halo subdwarfs and their expected magnitude/color 
distribution, based on the existing census. }
\item{Calculate the distance range over which LSST  will be detecting M subdwarfs of different subtypes, what can be expected of their  proper motion distribution, and determine the distance range required to assemble  statistically significant samples of various metallicity classes (i.e. halo sub-populations). }
\end{itemize}

tasks completed: 


\subsubsection{Co-moving populations}

definition: members of well-defined clusters, 
young associations, and moving groups in the  Solar Neighborhood 

interested collaborators: 
Michael Liu, Keivan Stassun, Sebastien Lepine 

interface with other collaborations: 
Star 
Clusters 

objectives: 
\begin{itemize}
\item{complete 
MFs for YAs/MGs in VLM/BD regimes }
]tem{search 
for new YAs/MGs, 
associated with “isolated” stars or as completely new groups. }
\item{identify 
ultracool dwarf benchmarks over a broad range in age ($\sim$1 Myr to $\sim$1 Gyr) }
\item{build 
up sample of ultracool dwarf benchmarks (i.e. with well constrained ages, 
compositions, distances) }
\item{disk/jet/accretion 
properties of young group members (e.g., TWA) }
\end{itemize}


issues: 

\begin{itemize}
\item{
probability 
of cluster membership: CPM, $\pi$/Dest, phot/spec properties, UV/X-ray, 
variability?; are RVs necessary? }
\item{how 
to we obtained activity diagnostics to examine youth? (e.g., Halpha) }
\item{Will 
u photometry be sufficient to characterize UV emission excess? }
\item{disentangling 
multiple memberships (overlapping nearby groups) }
\item{what 
is lowest mass we can probe to in a given group? }
\item{mapping 
observations to physical properties (Teff, logg, masses, ages) }
\item{mass 
sensitivity limit for nearest GCs, older clusters (including age effects) }
\item{The 
young local pop will start to merge with ScoCen - how do we interface with the Star 
Cluster subgroup?}
\end{itemize}

tasks to do: 
\begin{itemize}
\item{ determine 
requirements on RV measurement for cluster association }
\item{
confirm 
and compile properties of nearby YAs/MGs/GCs }
\item{predict 
mass limits for various groups }
item{
characterize 
impact of disk/jet emission/envelope obscuration on LSST (+2MASS 
+WISE?) photometry, map color terms to physical measures}
\end{itemize}

tasks completed: 



\subsubsection{VLM Multiples}

definition: all multiples with a VLM/BD primary, including PMOs  

interested collaborators: 
Adam Burgasser,  Michael Liu

interface with other collaborations  

objectives: 
\begin{itemize}
\item{measure 
the wide separation binary fraction and separation distribution via CPM }
\item{measure 
the small separation binary fraction via astrometric binaries }
\item{detect 
eclipsing binaries; if sufficient number, measure binary fraction at close 
separations }
\item{ use 
these measurements to constrain separation and mass ratio distributions, binary 
fraction all as a function of mass/SpT }
\item{use 
astrometric and transit binaries to determine orbital distributions (eccentricity, 
mass ratio) }
\item{use 
transit binaries to test VLM/BD structure models }
\item{robustly 
determine higher-order fractions for LM/VLM/BDs }
\item{measure 
planetary companion fraction to VLM/BDs }
\end{itemize}

issues: 
\begin{itemize}
\item{what 
is the transit detection rate for various cadences? what cadence would optimize? }
\item{what 
are the astrometric requirements for CPM assessment as a function of {binary 
separation, distance, mass ratio}? }
\item{how 
can color/photometry winnow candidates for wide binaries? }
\item{how 
reliable is photometric information for constraining mass ratios? separate question 
for */*, */BD and BD/BD pairs }
\item{what 
constraints can wide VLM binaries have on DM distribution? }
\end{itemize}

tasks to do: 
\begin{itemize}
\item{ensitivity 
limits for these three cases }
\item{simulate 
discovery fraction and efficiency for each class of systems (simulations of 
population, detectability for eclipsing/astrometric/resolved binaries) }
\item{interface 
astrometric requirements with halo PM, parallax requirements }
\item{false 
positive simulation for wide binaries }
\item{RV 
followup of astrometric, transit binaries }
\item{ compile 
specific predictions of VLM formation models on binary stats (e.g., co-ejection 
for wide multiples) }
\item{ model 
wide separation limits due to DM substructuring, differential tests with massive, 
LM, \& VLM wide binary distributions}
\end{itemize}

tasks completed: 



\subsubsection{VLM/BD companions to stars}

definition: resolved LM/VLM/BD/PMO companions to stars 

interested collaborators: 
Michael Liu  

interface with other collaborations:  

objectives: 
\begin{itemize}
\item{measure 
the companion fraction at low q to various stellar types }
\item{characterize 
“BD desert” at wide separations }
\item{build 
up sample of ultracool dwarf benchmarks (i.e. with well constrained ages, 
compositions, distances) }
\item{test 
brown dwarf evolutionary models and model atmospheres using benchmarks }
\item{identify 
wide companions to planet-hosting stars (Kozai mechanism) }
\end{itemize}

issues: 
\begin{itemize}
\item{confirmation: 
CPM (limits), color, other? }
\item{how 
to deal with saturated primaries }
\item{ what 
is the companion volume sampled as a function of separation (distance effects)?}
\item{what 
is the minimum resolving angle? }
\item{sensitivity 
as a function of angular separation from a star and brightness? }
\item{how 
to account for orbital alignments in detectability, separation distribution }
\item{ defining robust search samples - PM-selected? }
\end{itemize}


tasks to do: 
\begin{itemize}
\item{simulate 
discovery fraction with various assumptions of companion fraction, separation 
distribution, orbital properties }
\item{CPM limits }
\item{define 
input search sample; resources to characterize primaries }
\item{contamination 
rate as a function of separation (w/ \& w/o CPM confirmation) }
\item{resources for follow-up }
\end{itemize}


tasks completed: 


\subsubsection{Magnetic, atmospheric and structural photometric variability in cool dwarfs}

definition: analysis of photometric variability associated with magnetic and atmospheric  phenomena (separating off eclipses/transits) 

interested collaborators: 
John Gizis,  Keivan Stassun 

interface with other collaborations: 
transients and variables working group  

objectives: 
\begin{itemize}
\item{characterize 
period distribution of VLM dwarfs }
\item{characterize 
flaring rate and flare energy distribution of M and L dwarfs }
\item{characterize 
spot properties: covering fraction, Tspot-Tphot }
\item{distinguish 
between magnetic emission mechanisms (spots/aurorae) }
\item{characterize weather-related phenomena }
\end{itemize}


issues: 
\begin{itemize}
\item{relation 
of (lower-precision) LSST results to (higher-precision) Kepler \& TESS results? }
\item{cadence 
necessary for typical flare and rotation rates }
\item{what 
follow-up is required to characterize flares and/or quiescent emission? }
\item{Will 
targets of opportunity be needed to catch flares in action (is there even time to do 
this? what can we learn from GRB community?) }
\item{ photometric 
precision required for detection of X\% spot/cloud var }
\item{ what 
different bands sample in terms of cloud and flare properties }
\item{how 
could multi-color photometry constrain properties of flare source }
\item{can 
multi-color photometry constrain cloud properties (e.g., grain size distribution)}
\item{follow 
up to directly measure magnetic activity indicators (RAVE?) }
\end{itemize}

tasks to do: 
\begin{itemize}
\item{ use 
a list/database of currently known VLMs within X pc, and use this as in input catalog 
to see what the cadence of these sources would be. }
\item{model 
flares in LSST photometric bands as function of flare temperature }
\item{predictions 
of “magnetic” variability for spot vs. auroral models }
\item{model 
cloud variability in LSST photometric bands as function of v sin i, covering 
fraction, Tcloud-Thole, viewing angle, latitude distribution, differential rotation }
\item{discovery 
rate of flares for a given flare rate/energy model and cadence }
\end{itemize}

tasks completed 


\subsubsection{White dwarfs}

definition: all degenerates, including those not identified by parallax  

interested collaborators: 
Sebastien Lepine 

interface with other collaborations: 
Galactic 
Structure and ISM  

objectives: 
\begin{itemize}
\item{identify and characterize halo WD population }
\item{characterize 
the luminosity function across populations }
\item{identify 
main-sequence + WD binaries }
\end{itemize}

issues: 
\begin{itemize}
\item{Are u-g/g-r colors sufficient to identify most field white dwarfs? }
\item{To 
what extent will the search for WDs have to rely on proper motions (i.e. reduced 
proper motion detection). }
\item{How 
efficiently can WD+MS binaries be identified based on color alone? }
\end{itemize}

tasks to do: 
\begin{itemize}
\item{Calculate the local WD density based on the current census, and estimate how many 
WD can potentially be detected by LSST as a function of magnitude. }
\item{Build 
expected color-magnitude distributions based on predicted luminosity functions. }
\item{Estimate 
proper motion detectability of halo WDs, how deep one needs to go to 
assemble a statistically significant sample.}
\end{itemize}

tasks completed: 



