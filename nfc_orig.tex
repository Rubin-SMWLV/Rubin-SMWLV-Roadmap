\section{LSST Near Field Cosmology Roadmap}

The LSST Near Field Cosmology subgroup focuses on using stellar substructure throughout the Milky Way
 and Local Volume as a probe of the underlying CDM cosmology. 
This includes searches for dwarf galaxies, tidal streams and other coherent features in 
phase space using the properties of individual stars. 
Examples include searches for ultra-faint galaxies based on overdensity of main sequence stars, 
searches for substructure using RR Lyrae variables, or searches for moving stellar groups using proper motions. 

Searches for Milky Way substructure will allow critical new constraints in several areas. 
A more complete census and map of the distribution of satellites around the Milky Way sheds 
light both on the formation of satellite systems and the nature of dark matter halos that are 
able to host such low-mass galaxies.
 Finding tidal debris beyond 100kpc and out to the virial radius of the Milky Way will open a 
window on the more recent accretion history of the Galaxy, as well as the accretion of objects 
on more radial orbits in particular. 
Debris in the outer Galaxy also promises a powerful new probe of the mass 
distribution around the Milky Way in regions of a dark matter halo that have been largely 
unexplored for any galaxy in the Universe. 


\subsection{Overview of LSST NFC Science Goals}
\begin{itemize}
\item Mapping Milky Way substructure, including dwarf galaxies and stellar streams using main 
sequence stars out to $\sim$300kpc 
\item Map of Milky Way substructure using more luminous tracers (BHB, RRL) out to 800 kpc 
\item Search for substructure in tangential velocity field out to 25 kpc 
(60 km s$^-1$ precision). 
\end{itemize}

\subsection{Scientific and Technical Issues}

We discuss below issues that go beyond the LSST design specifications. 
\begin{enumerate}
\item{StarGalaxy Separation: For apparent magnitudes beyond $r\sim22$, the number of unresolved
 galaxies relative to stars rises dramatically. StarGalaxy separation is critical and 
improvements in StarGalaxy separation will likely have 
the most significant influence on the degree to which the above
 science goals are achieved. 

StarGalaxy separation will be done using both color and morphological criteria. 
There is general consensus that these two type of criteria remain separate in the 
LSST catalogs, allowing users to combine them in different ways for different science goals. 
combined for specific science goals. There is a desire to see as many measures relevant to
StarGalaxy separation as possible. Preferably as a probability, or some index like ($\chi^2, \nu$)
 that is transformable to probability. }

\item{AllSky Query Capability: Search for Milky Way substructure, particularly tidal streams, 
may be one of the few sciences cases that requires full survey data access and cannot be 
accomplished on smaller data chunks. This is unlikely to be possible in SQL on LSST servers. 
An SDSSLike Casjobs interface is both necessary and sufficient for this. }

\item{Variable Stars: The NFC group is primarily interested in RR Lyrae, but additional variable 
stars include W UMa stars, $\delta$Scuti, Miras and Cepheids are of interest. 
There was general agreement that current LSST cadance is sufficient for these searches.
 Important flags to search for variable stars include the Stetson J, K, and L indices, as well as the
 median magnitude, skewness, kurtosis, von Neumann index, and linear fits to magnitudes vs. time.}

\item{Flags for diffraction spikes, ghosts, etc: 
Searches for substructure depend critically on photometric uniformity. 
Need to have a catalog flag indicating the degree to which a source may be a 
product of (or affected by) diffraction spikes or ghosts. 
It is important that these flags are provide for each data epoch.}

\item{Proper Motion Flags: 
Both straightline measurements as well as with parallax folded in, with a 
$\chi^2$ for each. 
This would be good both to identify nearby stars as well as improve 
reliability of the proper motions of distant stars with too many fitting parameters. }

\item{Completeness Estimates: Many NFC science goals will require estimates of the sample
 completenesses as a function of magnitude for point sources in each color, at every position on the sky.
 This involves artificial star tests over an appropriate range of seeing, crowding, 
and for the appropriate number of exposures that go into each data release's average 
photometry. This will be particularly true for coadded data products.}
\end{enumerate}

