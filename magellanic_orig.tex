\section{Towards a Roadmap for LSST Magellanic Clouds Science}
Knut Olsen, with direct contributions from Nitya Kallivayalil, Beth Willman, Doug  Geisler, and Ben Williams  

\subsection{Introduction} 
The LMC and SMC are the nearest large galaxies to the Milky Way and represent  fundamental testbeds for investigations related to many stellar astrophysical topics,  from studying the star formation process, to the physics of variable star populations,  to the physics of galaxy interactions, and to calibrating the cosmological distance  scale.    

The Clouds are some of the bestGstudied galaxies in the Universe, yet nevertheless  consistently have the power to surprise.  A recent surprise is that the Clouds, which  have long been thought of as closely bound satellites of the Milky Way, appear to be  on first infall, or at most have made a small number of close passages, based on  proper motion measurements (Kallivayalil et al. 2006a,b; Besla et al. 2007, Piatek et  al. 2008, Kallivayalil et al. 2013).  We have also seen evidence that the LMC has  accreted a significant mass in stars from the SMC (Olsen et al. 2011), which may  indicate that the LMC and SMC collided directly with each other ~100 Myr ago  (Besla et al. 2012), a discovery that relied on the proper motion work.  The HI tidal  streamers that connect the LMC to the SMC and Magellanic Stream are kinematically  linked to the accreted SMC stars, and thus may have formed in this collision; the  streamers converge on the location of 30 Doradus in the LMC (Nidever et al. 2008),  indicating that 30 Doradus may be an interactionGinduced star formation event.   Moreover, the stellar debris left behind by the SMC could naturally explain the  microlensing event rate (Besla et al. 2013) seen towards the LMC by e.g. MACHO and  OGLE.  

The Magellanic Clouds are also now known to extend to much larger radius than  known before, through several pencilGbeam studies.  Majewski et al. (2009) found  spectroscopicallyGconfirmed KGgiant LMC stars at large distances (R $\sim$ 22 or 19 kpc),  while Saha et al. (2010) found that the LMC exponential disk extends past 12 disk  scale lengths.  In the SMC, Nidever et al. (2011) traced K giants to $\sim$11, with  possible evidence for a stellar halo.  These studies are based on exploration of 1  of the relevant area of sky.  There is thus a vast area that remains to be explored for  hidden Magellanic Clouds structure, with implications both for understanding the  history of the Clouds and the formation of the Galactic halo.   

There are many other recent exciting discoveries in the Magellanic Clouds that have  not been mentioned here.  The point of mentioning the few above is to demonstrate  that 1) in the Clouds, astrophysical problems are often connected across broad areas  of study, 2) Magellanic Cloud populations cover a very large area of sky, and 3)  highly precise measurements, such as produced by the work on LMC and SMC  proper motions, have the power to fundamentally alter accepted understanding.  

\subsection{LSST:relevant Magellanic Cloud Science Objectives}
 Magellanic Clouds science potentially covers a very broad range of topics.  Some of  the topics that the MC working group suggested of being of greatest interest are:  
\begin{enumerate}
\item{Mapping the Magellanic periphery.  Magellanic Cloud populations are now  known to extend out to ~20 degrees from the parent galaxies.  Whether  these populations are related to stellar halos, extended disks, or tidal debris  is not known.  The HI Magellanic Stream extends over 200 degrees of sky  (Nidever et al. 2010).  LSST will excel at finding stellar populations down to  very low effective surface brightness through deep photometry of individual  stars.  It will also provide a vast sample of RR Lyrae to trace potential  structures in three dimensions.  }
\item{Proper motions and internal motions.  The most precise proper motions in  the Clouds so far are those based on HST observations of narrow fields  centered on background QSOs (Kallivayalil et al. 2006).  LSST has the  potential to deliver precise proper motions over large swaths of the Clouds.   Given the complexity in the internal kinematics, these would be a boon for  piecing together the effects of the LMC and SMC recent interaction history.   As a demonstration of what is potentially possible, CasettiGDinescu et al.  (2012) found a significant proper motion difference between the  spectroscopically identified accreted SMC stars in the LMC and the general  LMC background, using groundGbased measurements.  }

\item{The star formation and chemical abundance history of the Magellanic Clouds.   Linking detailed knowledge of these to detailed knowledge of the interaction  history of the Clouds will be a big step forward in understanding what has  driven the evolution of the Clouds’ stellar populations.  }

\item{ Stellar variables.  The Magellanic Clouds contain the full array of known  variable stars, with the big advantage that their distances are well  constrained.  They also give important insight into variable star physics  through their low metallicities compared to the Milky Way.  Important  variable star classes are regular variables such as RR Lyrae, irregular  variables for which the Clouds provide complete statistical samples of events  such as flares, variables which are discovered at wavelengths other than  optical, such as xGray binaries, and stars which experience eclipses by other  stars or, potentially, planets.  }

\item{Star clusters.  The Magellanic Clouds are thought to contain thousands of star  clusters, only a fraction of which have been discovered and cataloged.  Star  clusters are very useful as precise markers in age and metallicity in the  history of star formation in galaxies, and critical for understanding questions  related star cluster dissolution and star formation mode.  LSST has the  potential to provide a complete catalog of Magellanic Cloud clusters to great  depth, including clusters at large radius from the parent galaxies.  }
\end{enumerate}

\subsection{Scientific and technical issues}
 The first big scientific issue is to identify those science cases that are uniquely  enabled by LSST.  In the case of the Magellanic Clouds, the existence of other largeG 
scale optical surveys such as the Magellanic Clouds Photometric Survey (Zaritsky et  al. 2000), DENIS, and the Dark Energy Survey, as well as the existence of wideGfield  cameras such as DECam, can give the impression that the science cases outlined can  be done by other means.  While it is certainly true that past and current surveys are  working on the problems above, LSST has the potential to open qualitatively new  discovery spaces through:  
\begin{enumerate}
\item{ Deep, multiepoch coverage of nearly the entire southern sky.  We have  
already shown that the search for Magellanic Cloud populations is one that  potentially covers thousands of square degrees.  While DES and the DECam  survey SMASH (Nidever et al. 2013) will be potential bonanzas for the study  of the Magellanic periphery, they probe only to g~24, the single'epoch depth  of LSST, and have limited time information.  LSST stacked photometry will go  several magnitudes deeper, yielding the potential to discover structure with  extremely faint equivalent surface brightness, and allow for the use of  variable stars as three dimensional probes of structure.  }

\item{Precise proper motions.  Much of our recent advance in revealing the  complexity of the Magellanic Cloud interaction history stems from the factor  of 10 improvement in proper motion that the HSTGbased work provided.  If  enough attention is given to the Clouds in the LSST cadence, LSST has the  potential to reveal the internal motions of the Clouds in unprecedented detail.  }

\item{Precise photometry.  All of the work described above depends on  photometric measurements.  Given the scale of typical PIGled projects and  calibration efforts, it is rare for projects to achieve <2  photometric  accuracy in the absolute sense.  The LSST science requirement is to deliver  1  absolute photometric accuracy in single images, with a stretch goal of  0.5 .  If these numbers can be achieved for the Clouds, and in particular  improved upon by averaging multiple measurements, metallicities based on  photometry, ages from comparison to isochrones, and reddening  measurements all stand to improve by large amounts.  As demonstrated by  the phenomenal results from Kepler, the realm of high photometric accuracy  has the potential to deliver many surprises.  }
\end{enumerate}

  Given the discovery spaces that will make LSST unique for Magellanic Clouds  science, there are a number of technical issues that need to be addressed as part of a  roadmap for the Magellanic Clouds:  
\begin{enumerate}
\item{What cadence is needed for full Magellanic Clouds coverage to achieve the  desired gains in proper motion and photometric accuracy?  The main bodies  of the Clouds, in the current baseline, are not part of the WideGFastGDeep  survey, but are covered at reduced frequency.  
\begin{enumerate}
\item{To evaluate the astrometric needs, we can quantify what HST/GAIA  can do per single star, and what requirements that imposes on LSST  observing in the MCs.  Dave Monet has been doing a lot of work within  the context of the DAWG to come up with metrics that best define  astrometry errors. Nitya summarizes from Monet: we want to (a) give  
specific examples of good and bad parallax cadences, and also (b)  develop metrics that can predict good and bad cadences so we can  optimize. So far, the former has shown that there is a factor of 2  difference in recovered errors based on cadences that otherwise take  the same amount of telescope time, and also as far as (b) goes, we  don't have a good metric so far, and all the various metrics one could  come up with seem to be correlated.  }

\item{To evaluate the level of photometric accuracy that we achieve in the  Clouds and the resulting cadence requirement, we can use the DECam  observations that are getting underway to begin answering this  question.  }
\end{enumerate}
}
\item{What limit on the photometric and astrometric accuracy is crowding going to  impose on us?  
\begin{enumerate}
\item{Past observations and the ongoing DECam surveys can help answer  the question of the fundamental limit imposed by crowding.  }

\item{How crowded images will get handled by LSST DM is a separate  question, and will require interaction with the DM Stack with  comparisons to crowded field photometric packages such as  DAOPHOT.  }

\item{A related question is whether there is the potential for a limited set of  excellent seeing observations in the Clouds, and whether that would  adversely impact the global cadence.  }
\end{enumerate}
}
\item{What is the best approach to starGgalaxy separation, particularly important  for studying the Magellanic periphery?  
\begin{enumerate}
\item{The SMASH survey has begun to address this question to the level of  r~24, but the growth in the galaxy luminosity function at faint  magnitudes means that we’ll probably need a dedicated effort to test  this to the depths that LSST will achieve.  }
\end{enumerate}
}
\item How do we accommodate the need for artificial star tests?  
\item{What other datasets need to be crossGmatched against the LSST catalog for  
the project to succeed?  a. We can begin constructing a list pretty easily.  }
\item{Are there particular projects that would be compelling/easy to do early in  the survey, perhaps during commissioning?  }
\item{What are the typical database queries and data services that we will be  using?   
a. This will need to be explored. There are efforts underway to provide a  testbed for potential services that would be useful to interact with  here.  }
\end{enumerate}

