\section{Roadmap for Star Clusters Subgroup}

Kevin Covey, Jay Strader, Jason Kalirai, \& Doug Geisler for the subgroup 

\subsection{A short summary of LSST science related to star clusters }

Star clusters are a critical component of many of the science goals listed in the LSST Science Book (Ivezic et al. 2010). They offer unique environments where stars with a range of masses but similar ages and chemical compositions can be found. One class of studies aims to leverage these relatively simple stellar populations to understand the dependence of the stellar mass function and stellar evolution on metallicity and age (e.g., Sec. 6.5-6.6, 6.9-6.11; LSST Sci. Book). Their known properties and distances also motivate calibration work on many kinds of variable stars and transients (e.g., Sec 8.9; LSST Sci. Book). 

A separate class of studies is focused on using star clusters to trace the star formation history, chemical evolution, and galactic structure of the Milky Way, nearby galaxies (including the Magellanic Clouds), and even more distant galaxies extending beyond the Local Volume (e.g., Sec. 6.2-6.3, 6.6, 7.2, 7.3, 7.8, 7.10, 7.11; LSST Sci Book). 

\subsection{Critical technical elements for LSST star cluster science }

The unique advantage of LSST for studies of star clusters is the promise of deep, accurate, and homo- geneous multi-band photometry and astrometry for the entire southern sky. This directly enables both (i) an enormous sample of clusters and (ii) spatially complete data for each cluster, neither of which are generally possible in current datasets. 

This promise can only be realized if LSST delivers photometric and astrometric measurements that are uniform, or at least well-behaved, over the wide range of crowding conditions and variable foregrounds and backgrounds that can occur within a single cluster and across different Galactic components. 

Specifically, we identify the following critical technical elements as most relevant to star cluster science with LSST: 
\begin{itemize}
\item The impact of source crowding and homogeneous backgrounds on the precision and depth of LSST photometry and astrometry. 
\item The performance of LSST’s deblending and multi-epoch source association algorithms as a function of seeing and in crowded or nebulous fields. 
\end{itemize}

\subsection{Near-term investigations to retire risk for LSST star cluster science }

Over the next year, the core members of the Star Cluster Working Group will use simulated LSST data products and precursor datasets to test the expected performance of LSST. Specifically, we plan to: 
\begin{enumerate}
\item{Use the ImSim pipeline to generate synthetic LSST images of crowded fields. Use these to establish the quality of star cluster color-magnitude diagrams—and their utility for testing specific aspects of stellar evolution models—using LSST’s photometry and astrometry. Image morphology statistics will also be used to test star–galaxy separation. (J. Kalirai) }
\item{Obtain the best existing examples of overlapping 
ultra-deep ground-based starcluster photometry with HST images 
(e.g., NGC 5466; Beccari et al. 2013), using the “ground truth” provided by the HST catalogs to empirically assess the expected photometric uncertainties and incompleteness in LSST data. (J. Kalirai) }
\item{Calculate the proper motions uncertainties—both absolute and internal—that LSST will be able to return for typical Galactic star clusters, using both 10-year LSST baselines and longer HST to LSST baselines. These metrics will be compared to Gaia’s expected performance as a function of magnitude and crowding to delineate the opportunity space for LSST in this area. (J. Strader) }
\item{Acquire new, wide-field multi-epoch ugriz photometry of dense and nebulous clusters in the Galactic plane. Process these images with the LSSTstack to quantify photometric precision as a function of local background and assess the fidelity of source association among the epochs. (K. Covey) }
\end{enumerate}

The primary person responsible for each task is listed above, but we expect that much of the effort will be collaborative both within the subgroup and with members of other relevant subgroups. 



