\section{Research and Development}

\subsection{Introduction to Collaboration R\&D Priorities}

A number of priorities have been defined with the SMWLV collaboration, here we give an overview. The details of these issues are spelled out in further sections.

{\bf These are from the milkyway website, we have an updated set from the 2019 PCW breakout sessions.}

\begin{enumerate}

\item{Crowded field photometry/astrometry (Example action items: DECam precursor studies of Galactic bulge/plane to estimate the source density as a function of position; Use ImSim pipeline to create synthetic images of crowded fields and test LSST?s astrometry and photometry on resultant CMDs).}

\item{Star-galaxy separation (Example action items: HST archival project to generate a point source/extended source truth table; Use of SMASH survey of Magellanic Clouds to empirically benchmark impact of unresolved galaxies on LSB structure mapping; Multi-color deep wide-field surveys of known LSB systems to test star-galaxy separation algorithms; Use ImSim input catalog to test star-galaxy separation algorithms).}

\item{Photometric metallicity calibration (Example action items: Assemble relevant spectral and photometric datasets. Compare to existing efforts from SDSS, including Ivezic metallicities for FGK stars, and Mann, Lepine and West calibrations for M dwarfs. What are the ranges in mass/color where these relations are valid? Extend these applications to LSST, assuming a LSST-SDSS transformation is robust. Using archival spectra or PanSTARRS data, test how y-band photometry can improve the photometric metallicity estimate. How can time-series information be used to improve metallicity estimates (e.g., Miller, Richards, Bloom et al).}

\item{y4 filter calibration (Example action items: Ground based/space based y4 investigations; Can PanSTARRS experience be leveraged for y-band calibration? How does varying the water column change the spectral response. Simulate the effect of a varying water column on the y4 filter using existing spectra of FGKM stars.).
Cadence Planning (Example action items: Build the case for Magellanic Clouds cadence; Develop the science and technical case for MW disk and bulge cadence; Develop and test cadence metrics for astrometry).}

\item{Stellar variability classification (Example action items: Define classification metrics required for your specific science goals, see example in Near Field Cosmology roadmap; Develop light curve classifiers + period finders and test on existing datasets [e.g. LINEAR]; coordinate with Transients science collaboration and tools from EB factory general purpose light curve classifier and intelligent period finder; DECam Bulge survey).}

\item{Subgroup specific technical needs (Action items: As defined by individual collaboration members and Working group leads, e.g. photometry of nebulae.).}

\item{Theoretical predictions for LSST Milky Way science (Example action items: As defined by individual collaboration members and Working group leads, e.g., simulations that make predictions for LSST; coordinate with Justin Read?s Gaia challenges).}

\end{enumerate}


\subsection{Update from 2019 PCW}

Here are some important updates from the LSST Project and Community Workshop held earlier this month in Tucson:

\begin{enumerate}

\item{Crowded Field Photometry: The project is making a plan to process crowded stellar fields with a special deblender, i.e., a DAOPhot-like system integrated into the LSST DM pipeline, and not the same deblender being developed to deal with galaxies at high galactic latitude. They believe an in-house DM solution has significant advantages over using existing external codes. A document (plan) explaining their approach is expected around September 1. We will have two months to provide feedback. [See Task Forces below]}

\item{New "Task Forces" for 2019-2020:  We had a meeting of SMWLV members. We decided to form a number of task forces which have a defined mission and finite lifetime.  A number of volunteers, but we badly need more.}

\end{enumerate}

\subsubsection{Proposed task forces}

CADENCE SIMULATIONS: Tasks: Review existing cadence metrics and identify new metrics. Work with LSST MAF group to implement metrics. Complete this within 4 months. After feedback from our collaboration, analyze results of simulations.  Volunteers: Olsen, Clarkson, Leo, Marcio,

CROWDED FIELD: Read and comment on DM crowded field photometry plan by November. Volunteers: Clarkson, Nidever,Adriano Pieres 

COMMISSIONING: Review LSST official document on commissioning tests. Lead discussion of tests that we feel are missing. Identify participants to directly participate in commissioning, early access sprints, data preview sprints. Many timescales, but reading and planning can occur in next six months. We anticipate most work is in 2020-2021. Volunteers: Olsen. 

ASTROMETRY: Investigate Differential Color Refraction (DCR). Identify data and develop proposal if necessary. Volunteers: D. Monet, Jullio Camargo

CALIBRATION: Investigate water bands and calibration (y) filter. New calibration data expected in spring 2020, so start then. (Should also reach out to supernovae people). Volunteer: Pat Boeshaar.

DATA CHALLENGE: Plan sprint or data challenges of our own to do science. Volunteer: Clarkson.

Other task forces are welcome. The first three were considered the most urgent and important.  Please volunteer! 


\subsubsection{Crowded field photometry}

\subsubsection{Star-galaxy separation}

\subsubsection{Photometric metallicity calibration}

\subsubsection{y4 filter calibration}

\subsubsection{Stellar variability calibration}

\subsubsection{Subgroup specific technical needs}

\subsubsection{Theoretical predictions for LSST SMWLV science}







